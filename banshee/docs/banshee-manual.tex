% Copyright (c) 2000-2005
%      The Regents of the University of California.  All rights reserved.
%
% Redistribution and use in source and binary forms, with or without
% modification, are permitted provided that the following conditions
% are met:
% 1. Redistributions of source code must retain the above copyright
%    notice, this list of conditions and the following disclaimer.
% 2. Redistributions in binary form must reproduce the above copyright
%    notice, this list of conditions and the following disclaimer in the
%    documentation and/or other materials provided with the distribution.
% 3. Neither the name of the University nor the names of its contributors
%    may be used to endorse or promote products derived from this software
%    without specific prior written permission.
%
% THIS SOFTWARE IS PROVIDED BY THE REGENTS AND CONTRIBUTORS ``AS IS'' AND
% ANY EXPRESS OR IMPLIED WARRANTIES, INCLUDING, BUT NOT LIMITED TO, THE
% IMPLIED WARRANTIES OF MERCHANTABILITY AND FITNESS FOR A PARTICULAR PURPOSE
% ARE DISCLAIMED.  IN NO EVENT SHALL THE REGENTS OR CONTRIBUTORS BE LIABLE
% FOR ANY DIRECT, INDIRECT, INCIDENTAL, SPECIAL, EXEMPLARY, OR CONSEQUENTIAL
% DAMAGES (INCLUDING, BUT NOT LIMITED TO, PROCUREMENT OF SUBSTITUTE GOODS
% OR SERVICES; LOSS OF USE, DATA, OR PROFITS; OR BUSINESS INTERRUPTION)
% HOWEVER CAUSED AND ON ANY THEORY OF LIABILITY, WHETHER IN CONTRACT, STRICT
% LIABILITY, OR TORT (INCLUDING NEGLIGENCE OR OTHERWISE) ARISING IN ANY WAY
% OUT OF THE USE OF THIS SOFTWARE, EVEN IF ADVISED OF THE POSSIBILITY OF
% SUCH DAMAGE.

%\documentclass{sig-alternate}
\documentclass[10pt]{article}
%\usepackage{fullpage}
\usepackage{theorem}
\usepackage{verbatim}

\newcommand{\bane}{\textsc{Bane}}
\newcommand{\banshee}{\textsc{Banshee}}
\newcommand{\ibanshee}{\textsc{IBanshee}}
\newcommand{\contra}[1]{\overline{#1}}
\newcommand{\file}[1]{\texttt{\textbf{[#1]}}}
\newcommand{\cmd}[1]{\texttt{\textbf{#1}}}
\newcommand{\var}[1]{\mathcal{#1}}
\newcommand{\id}[1]{{\it #1\/}}
\newcommand{\bra}[1]{\langle {#1} \rangle}
\def\kw#1{\hbox{\tt #1}}

{\theorembodyfont{\rmfamily}
\newtheorem{example}{Example}

\title{Banshee Tutorial and Overview} \author{John Kodumal
  (University of California at Berkeley) \\ \texttt{jkodumal@cs.berkeley.edu}}

\begin{document}
\maketitle

\vspace*{2in}
\noindent
This documentation is copyright (c) 2005 The Regents of the
University of California.  \banshee{} is distributed without any
warranty.  See the notice in Appendix~\ref{app-copyright} for full
copyright information.

\newpage
{\small \tableofcontents{}}
\newpage

\section{Introduction}
\label{sec-intro}

\banshee{} is a highly optimized toolkit for constructing
constraint-based program analyses. \banshee{}, like its predecessor
\bane{} \cite{aiken:tic98}, is based on \emph{mixed constraints},
allowing users to design their own ad-hoc analysis formalisms by
mixing standard, well understood constraint languages
\cite{fahndrich:sas97}. \banshee{}'s main innovation is its use of an
analysis specification language to \emph{specialize} the constraint
resolution engine for specific program analyses. This approach yields
several distinct advantages over previous toolkits:
\begin{itemize}
\item \textbf{Cleaner user interfaces.} Given an analysis
  specification, \banshee{} creates a compact interface tailored to
  the analysis.
\item \textbf{Better type safety.} Constraints are typically subject
  to a set of \emph{well-formedness} conditions. In previous toolkits,
  these conditions were checked dynamically. In \banshee{} more of
  these conditions are checked statically, reducing the possibility of
  run-time errors.
\item \textbf{Improved performance.} \banshee{} applications realize a
  performance gain as an effect of reducing these dynamic checks and
  generating specialized C code.
\item \textbf{Better extensibility.} The system can easily be extended
  to handle new constraint formalisms.
\end{itemize}

Other novel features in \banshee{} include an efficient implementation
of persistence (constraint systems can be quickly saved/loaded), and a
backtracking capability (constraint systems can be rolled back to any
previous state) that can be used for "poor-man's" incremental analysis
\cite{kodumal:sas05}.

There are a number of ways for an analysis designer to use
\banshee{}. These will be explained in detail later, but we provide an
overview here:
\begin{itemize}
\item \textbf{As an interpreter}
  (Section~\ref{sec-ibanshee}). \banshee{} comes with an interpreter,
  \ibanshee{} \file{bin/ibanshee.exe}. \ibanshee{} allows the designer
  to interact with \banshee{} by typing in constraints directly. This
  is a handy tool for learning about set constraints. It's also an
  easy way to interface with \banshee{}, especially if the client is
  written in a language other than C.
\item \textbf{As a generic C library}
  (Section~\ref{sec-nonspec}). Here, the analysis designer simply
  links their client to a C library \file{engine/libnsengine.a}. This
  library has a generic (nonspecialized) interface
  \file{engine/nonspec.h}. This approach forgoes the benefits of
  specialization, but allows new constructors to be defined
  dynamically.
\item \textbf{As a specialized C library}
  (Section~\ref{sec-spec}). Here, the analysis designer writes a short
  specification file describing the constraint language needed for her
  analysis. The \banshee{} code generator \file{bin/banshee.exe} is
  invoked on this file and produces a specialized constraint solver
  interface. The analysis client and the specialized interface are
  then linked with a backend C library \file{engine/libengine.a}.
\item \textbf{As a context-free language reachability engine}
  (Section~\ref{sec-dyckcfl}).  \banshee{} also includes a wrapper API
  \file{dyckcfl/dyckcfl.h} that implements a very efficient reduction
  from a restricted class of context-free language reachability to set
  constraints. Many program analysis problems fall into this
  restricted class, so we have implemented an API to express these
  problems.
\end{itemize}

We are also working on foreign function interfaces for
\banshee{}. These would simplify writing analyses in languages other
than C. Currently, there is a Java interface (via JNI) for the
context-free language reachability API. We plan to have Java and
O'Caml interfaces for the entire \banshee{} system soon.

\section{Installation}
\label{sec-install}

\subsection{Prerequisites}
Here is a list of the software packages required to build \banshee{}:

\begin{itemize}
\item Objective Caml version 3.0.8.2 or later (available at
  http://caml.inria.fr)
\item GCC, the Gnu Compiler Collection, version 3.3 or later
\item Python, version 2.3.3 or later 
\item indent 
\item etags
\item flex
\item bison
\end{itemize}

With the exception of Objective Caml, these utilities should be
standard.

\subsection{Compiling \banshee{}}

Assuming all prerequisites are installed, you can compile banshee by
typing \texttt{make all check} or \texttt{gmake all check} at the
command line in the top level of the source tree. If everything
succeeds, make will report \texttt{SUCCESS: All check targets passed.}
and the following targets will be built:

\begin{itemize}
\item \file{bin/banshee.exe}, the \banshee{} code generator, and
  \file{engine/libengine.a}, the backend constraint library. These
  targets are required to use \banshee{} as a specialized engine.
\item \file{engine/libnsengine.a}, the nonspecialized constraint
  library. This target is required to use \banshee{} as a generic
  constraint library or context-free language reachability engine.
\item \file{bin/ibanshee.exe}. This target is \ibanshee{}, the
  \banshee{} interpreter.
\item \file{dyckcfl/dyckcfl.o}. This object file is required (in
  addition to \file{engine/libnsengine.a}) to use \banshee{} as a
  context-free language reachability engine.
\end{itemize}

\cmd{make} will also build the user manual (this document) a suite of
small test applications (see \file{tests/README}) and the points-to
analysis application (see \file{cparser/README}).

Here is a summary of the most useful \texttt{make} targets:

\begin{itemize}
\item \cmd{make all} will build everything.
\item \cmd{make check} will verify your build.
\item \cmd{make banshee} will build banshee without building any test
  applications.
\item \cmd{make docs} will just build the documentation.
\item \cmd{make points-to} will build the points-to analysis
  application and any dependencies.
\item \cmd{make ibanshee} will build \ibanshee{} and any dependencies.
\end{itemize}

\banshee{} should build and run on Linux, FreeBSD, Windows (with
cygwin) and MacOS X. On FreeBSD, make sure to use \texttt{gmake}
instead of \texttt{make}.

\section{Tutorial}
\label{sec-tutorial}

In this section, we give a brief tutorial and gentle introduction to
set constraints. We'll explain the basic formalism and demonstrate how
to express some simple problems in \ibanshee{}. For those who have
some familiarity with set constraints, this tutorial covers the simple
term-set model of set constraints. The complete \banshee{} system
supports a much richer model--- we refer the interested reader to
\cite{fahndrich:sas97,fahndrich:thesis} for further details.

\subsection{Expressions}

We'll begin our discussion with a few definitions. A \emph{ranked
  alphabet} is a finite set of constructors and a function
$arity(...)$ that maps each constructor to a nonnegative integer (the
\emph{arity} of the constructor). We use $c,d,e,\ldots$ to range over
constructors. $0$-ary constructors are called \emph{constants}. We
also assume that there is a special set of constants called
\emph{variables} which are disjoint from our ranked alphabet. We'll
use $\var{X}, \var{Y}, \var{Z}, \ldots$ to range over variables.

Think of a ranked alphabet as a set of building blocks for making
\emph{expressions} (or trees). 

\begin{example}
Suppose our alphabet is the set of constructors:
\[
f,g,c
\]
with $arity(f) =2$, $arity(g) = 1$, $arity(c) = 0$. This alphabet
defines a set of expressions over $f,g,c$, which we can build in the
natural way. Examples of expressions are $f(\var{X},\var{Y}),
f(g(c()),c())$, $f(c(),c())$, $g(f(c(),c()))$. The last three
expressions are \emph{ground terms}, which are expressions that don't
contain any variables. As shorthand, we'll often use $c$ instead of
$c()$ for constants.
\end{example}

More formally, the set of expressions over a ranked alphabet and set of
variables is defined inductively:
\begin{itemize}
\item $\var{X},\var{Y},\var{Z},\ldots$ are all expressions
\item $c$ is an expression if $arity(c) = 0$
\item $f(e_1,\ldots,e_n)$ is an expression if $arity(f) = n$ and
  $e_1,\ldots,e_n$ are expressions
\end{itemize}

\subsection{Expressions in \ibanshee{}}

Before proceeding further, we'll pause and explain how to declare
constructors and variables in \ibanshee{}. Both constructors and
variables must be declared before use. \ibanshee{} constructors must
begin with an upper or lower case letter, followed by a string of
letters, numbers, or underscores. Declaring a constant is
simple \footnote{The \texttt{[0] >} is \ibanshee{}'s prompt. The line
  immediately following a line starting with a prompt is \ibanshee{}'s
  output. If you want to try these examples out, just type the
  remainder of each line after the prompt.}:

\begin{verbatim}
[0] > c : setIF
constructor: c
\end{verbatim}
Here, we've created a constant $c$. The colon followed by
\texttt{setIF} is a \emph{sort} declaration in \ibanshee{}. Think of a
sort as a type (like int or float in a conventional programming
language). Since we're focusing on set constraints for now, we'll
defer our explanation the other sorts that \banshee{} provides. For
now, everything we make will be of sort \texttt{setIF}.

Declaring an $n$-ary constructor is slightly more involved, because we
need to tell \ibanshee{} the constructor's arity. We do so with a
comma separated list of sort declarations:

\begin{verbatim}
[0] > f(+setIF,+setIF)
constructor: f
\end{verbatim}

This declares a binary constructor $f$. Notice the plus signs before
the sort declarations in this example. These are \emph{variance}
declarations. We'll defer discussion of variance declarations until
later. For now, we'll just restrict ourselves to \emph{positive}
variances, which are specified with \texttt{+}.

Variables are declared just as constants. However, \ibanshee{}
syntactically distinguishes variables from constructors by forcing you
to begin each variable name with a tick (\texttt{'}). So for example:

\begin{verbatim}
[0] > 'x : setIF
var: 'x
\end{verbatim}

declares a variable called \texttt{'x}.

Note that \ibanshee{} only asks you to declare constructors and
variables. There is no need to declare other terms explicitly. In
other words, once you've defined \texttt{f} and \texttt{'x} as in the
preceeding examples, you can refer to an expression like
$\texttt{f('x,'x)}$ without declaring it first. Of course, we have yet
to explain how to do anything useful with expressions... so without
further ado, let's talk about constraints!

\subsection{Constraints}

Constraints are inclusion relations between expressions. A constraint
is a relation of the form $e_1 \subseteq e_2$, where $e_1$ and $e_2$
are expressions. A system of constraints is a finite conjunction of
constraints. A solution $S$ of a system of constraints is a mapping
from variables to sets of ground terms such that all inclusion
relations are satisfied. Let's see an example:

\begin{example}
Suppose that we have $f,g,c$ as constructors with $arity(f) = 2$,
$arity(g) = 1$, and $arity(c) = 0$ as before, along with variables
$\var{X}$ and $\var{Y}$. Now consider the following system of
constraints:
\[
f(\var{X},g(\var{X})) \subseteq f(\var{Y},\var{Y})
\]
\[
c \subseteq \var{X}
\]
A solution to this system of constraints is $S$ where $S(\var{X}) =
{c}$, $S(\var{Y}) = {c,g(c)}$. You can see that this solution
satisfies the inclusion constraints by substituting in for the
variables.
\end{example}

A system of constraints may have no solutions, one solution, or many
solutions. Often we will want the least solution or the greatest
solution. For the simple model of constraints explained here, the
least and greatest solutions will be unique (assuming there are any
solutions at all!). In the previous example, the solution $S$ is
least.

\subsection{Constraints in \ibanshee{}}

Creating a system of constraints in \ibanshee{} is
straightforward. After defining a set of constructors and variables,
you can build expressions over those constructors and variables and
add constraints. We'll continue with an \ibanshee{} program that
corresponds to the previous example:

\begin{verbatim}
[0] > f(+setIF,+setIF):setIF
constructor: f
[0] > c : setIF
constructor: c
[0] > g(+setIF) : setIF
constructor: g
[0] > 'x : setIF
var: 'x
[0] > 'y : setIF
var: 'y
[0] > f('x,g('x)) <= f('y,'y)
[1] > c <= 'x
\end{verbatim}

Now that we have defined our constraint system, we'll probably want to
inspect the solutions of the constraints. We can do so using
\ibanshee{}'s query commands. \ibanshee{} commands always begin with
\texttt{!}. The first command is \texttt{!tlb}, which stands for
\emph{transitive lower bounds}. This command allows you to read off
the least solution of the constraints:

\begin{verbatim}
[2] > !tlb 'x
{c}
[2] > !tlb 'y
{c, g('x)}
\end{verbatim}

With a little investigation we see that this is exactly the least
solution. But why doesn't \ibanshee{} explicitly compute it? In other
words, why doesn't \ibanshee{} report that \texttt{'y} maps to the
ground set \texttt{\{c,g(c)\}}? Another example is in order. 

\begin{example}
Consider the following constraint system: 
\[
g(\var{X}) \subseteq \var{X}
\]
\[
c \subseteq \var{X}
\]
There is a solution set for $\var{X}$, namely $\var{X} = \{c, g(c),
g(g(c)),\ldots\}$. However, there are no finite solutions for
$\var{X}$.
\end{example}

Thus, it may not be possible for \banshee{} to explicitly enumerate a
set of ground terms as a solution for a given variable. A little more
machinery is required to understand solutions and fully explain what
the \texttt{!tlb} command computes.  Briefly, it turns out that each
set variable describes a \emph{regular tree language} and that
solutions of set constraints can be viewed as collections of regular
tree grammars. The solution for $\var{X}$ in the previous example
corresponds to the regular tree grammar
\[
X \Rightarrow c
\]
\[
X \Rightarrow g(X)
\]
and we see that the \texttt{!tlb} command essentially returns the
right-hand sides of the grammar productions corresponding to the
solution for a given variable. A complete discussion is beyond the
scope of this tutorial; we refer the interested reader to
\cite{heintze:thesis} for more details.

\subsection{Variance}

Recall the notation we used to define the binary constructor
\texttt{f} in \ibanshee{}:

\begin{verbatim}
[0] > f(+setIF,+setIF):setIF
constructor: fun
\end{verbatim}

Let's explain the meaning of the \texttt{+} syntax. We'll start with a
simple example:

\begin{example}
Suppose we have a binary constructor $pair$ that models sets of
pairs. Consider the sets $\var{X} = \{1,2\}$ and $\var{Y} =
\{3\}$. Then $pair(\var{X}, \var{Y}) = \{pair(1,3),pair(2,3)\}$. Now,
suppose we add $4$ to the set $\var{Y}$. Intuitively, $pair(\var{X},
\var{Y})$ should grow to
$\{pair(1,3),pair(2,3),pair(1,4),pair(2,4)\}$, so
$pair(\var{X},\var{Y})$ grows as $\var{Y}$ grows. Notice that
$pair(\var{X},\var{Y})$ grows as $\var{X}$ grows, as well. In general,
the set $pair(e_1,e_2)$ grows if either of $e_1$ or $e_2$ grows.
\end{example}

We say that a constructor $c$ has \emph{positive} variance in its
$i$th position if the set $c(\ldots,e_i,\ldots)$ grows as $e_i$
grows. In the preceeding example, the constructor $pair$ is positive
in both positions. This is the meaning behind the \texttt{+} syntax in
ibanshee{}: it allows the user to tell \ibanshee{} that a constructor
has positive variance in some position.

The correct variance depends on what the constructor is intended to
model. Let's look at another example:

\begin{example}
Suppose we have a binary constructor $fun$ that models sets of
functions (the first argument in the function domain, the second
argument the function range). Suppose we also have sets $int$
containing the integers, and $real$ containing the real numbers, with
$int \subset real$. Now consider the sets $fun(int,int)$ and
$fun(real,int)$. Which set is larger? Consider any function in the set
$fun(real,int)$. Such a function expects a $real$ argument. However,
any $int$ is also a $real$, so any function in $fun(real,int)$ is also
in the set $fun(int,int)$. Therefore, $fun(real,int) \subseteq
fun(int,int)$. In this case, if $e_1$ shrinks, $fun(e_1,e_2)$ grows.
\end{example}

We say that a constructor $c$ has \emph{negative} variance in its
$i$th position if the set $c(\ldots,e_i,\ldots)$ shrinks as $e_i$
grows. In the preceeding example, the constructor $fun$ is negative in
its first argument. In \ibanshee{}, this can be specified by
preceeding a sort name with a \texttt{-} sign:

\begin{verbatim}
[0] > fun(-setIF,+setIF):setIF
constructor: fun
\end{verbatim}

With a little thought, it's easy to see that the range of a function
should have positive variance.

Another way of looking at variances is as follows: if $c(\ldots, e_i,
\ldots) \subseteq c(\ldots, e_i', \ldots)$, with $c$ positive in $i$,
then $e_i'$ contains \emph{at least} the set $e_i$ ($e_i \subseteq
e_i'$). If $c$ is negative in $i$, then $e_i'$ contains \emph{at most}
the set $e_i$ ($e_i' \subseteq e_i$). There is actually a third kind
of variance, \emph{non} variance. If $c$ is non variant in $i$, then
$e_i'$ contains \emph{exactly} the set $e_i$ ($e_i = e_i'$). In
\ibanshee{}, you get non variance by using \texttt{=} in place of
\texttt{+} or \texttt{-}:

\begin{verbatim}
[0] > ref(+setIF,=setIF):setIF
constructor: ref
\end{verbatim}

\subsection{Sorts}

A \banshee{} sort is a combination of
\begin{itemize}
        \item a language of expressions,
        \item a constraint relation between expressions,
        \item a solution space,
        \item and a resolution algorithm
\end{itemize}

Here we give a brief description of the sorts available in \banshee{}.

\subsubsection{Sets in Inductive Form}

Up to this point, we've restricted ourselves to \texttt{setIF}
expressions. The \texttt{IF} stands for \emph{inductive form}, which
is a way of representing and solving set constraints efficiently
\cite{aiken:tic98}. The \texttt{setIF} sort provides inclusion
constraints between set expressions. The solution space is sets of
regular trees.

\subsubsection{Sets in Subtransitive Form}

\banshee{} also supports subtransitive form, an alternative algorithm
for solving set constraints \cite{heintze:pldi01}. Other than having a
different resolution algorithm, \texttt{setST} is identical to
\texttt{setIF}. 

\subsubsection{Terms}

The \texttt{term} sort provides equality constraints between
expressions. The solution space is regular trees. The resolution
algorithm is essentially Robinson's unification algorithm. 
  
The \texttt{term} sort also provides conditional equality constraints
\cite{steensgaard:popl96}. A conditional equality constraint $t_1
\leq t_2$ is satisfied if either $t_1 = 0$, or $t_1 = t_2$.

\subsubsection{Rows}

A \emph{row} is a finite map from strings to expressions. For each
sort $s$, \banshee{} provides a corresponding row sort $row(s)$ that
maps strings to expressions of sort $s$. Rows essentially model
\emph{records} with any combination of width and depth subtyping. Note
that \banshee{} does not support rows of rows.

\subsection{Mixed Constraints}

One of the most powerful features of \banshee{} is its support for
\emph{mixed constraints}. Mixed constraints permit constructors to
contain subexpressions of different sorts. This allows the user to
find the "sweet spot" between efficiency and precision when designing
an analysis. Mixed constraints are easy to use: simply define a
constructor that uses multiple sorts:

\begin{verbatim}
[0] > f(+setIF,=term):setIF
constructor: ref
\end{verbatim}

With \texttt{f} defined as above, it is possible to add constraints
between mixed expressions (expressions with subexpressions of
different sorts):

\begin{verbatim}
[0] > 'x : setIF
[0] > 'y : term
[0] > 'z : setIF
[0] > c : term
[0] > f('x, c) <= f('z,'y)
[1] > !ecr 'y
c
\end{verbatim}

\subsection{Backtracking}

\banshee{} has a feature called \emph{backtracking} that allows the
user to "roll back" the state of a constraint system at any point in
time. In \ibanshee{}, the number displayed at the prompt tells the
current "version" of the constraint system. Backtracking allows the
constraint system to be rolled back to any previous version, by
specifying the version number:

Let's work with the previous example:

\begin{verbatim}
[0] > f(+setIF,=term):setIF
constructor: ref
[0] > 'x : setIF
[0] > 'y : term
[0] > 'z : setIF
[0] > c : term
[0] > f('x, c) <= f('z,'y)
[1] > !ecr 'y
c
\end{verbatim}

Notice that the version number is incremented after the addition of
the constraint \texttt{f('x,c) <= f('z,'y)}. That means that the
constraint system's version prior to the addition of that constraint
is \texttt{0}, and the version after the constraint addition is
\texttt{1}. In \ibanshee{}, backtracking is accomplished by the
command \texttt{!undo [i]}, where \texttt{i} is the version of the
constraint system to backtrack to:

\begin{verbatim}
[1] > !undo 0
[0] > !ecr 'y
'y
\end{verbatim}

After the \texttt{!undo} command, the constraint system reverts to its
state just before the constraint was added. Since \texttt{'y} is
unconstrained, its equivalence class representative is itself.

\subsection{Persistence}

Constraint systems can be saved or loaded to disk. The complete
internal representation of a constraint system is saved, so after
restoration all operations are legal (including backtracking).

\section{Using IBanshee}
\label{sec-ibanshee}

\subsection{IBanshee Commands}

\begin{itemize}
\item \texttt{!help} Print the quick reference
\item \texttt{!tlb e} Print the transitive lower bounds of \texttt{e}
\item \texttt{!ecr e} Print the equivalence class representative of \texttt{e}
\item \texttt{!undo i} Roll back the constraint system to its state at time \texttt{i}
\item \texttt{!trace i} Set the trace level to depth \texttt{i}. This will print constraints for recursive calls to the constraint solver up to \texttt{i} levels deep. Helpful for debugging.
\item \texttt{!quit} Exits iBanshee
\item \texttt{!save "filename"} Saves the current constraint system
\item \texttt{!load "filename"} Loads the constraint system saved using the \texttt{save} command
\item \texttt{!rsave} Saves the current constraint system (the filename is currently hardcoded) using region serialization. This is much faster than using \texttt{!save}.
\item \texttt{!rload} Load the constraint system saved with
  \texttt{rsave}.
\item \texttt{!exit} Exits iBanshee
\end{itemize}

\section{Using the Nonspecialized Engine API} 
\label{sec-nonspec}

In addition to the interactive constraint solver \ibanshee{},
\banshee{} includes a nonspecialized C library
(\file{engine/libnsengine.a}) that provides a direct interface to the
constraint solver.

To use \banshee{} as a nonspecialized library, these steps should
be followed:

\begin{itemize}
\item The analysis designer should include the header file
  \file{engine/nonspec.h} in their application. In general, this
  should be the only \banshee{} header included.
\item The client should call \texttt{nonspec\_init} before calling any
  other \banshee{} functions.
\item The client should be linked with the libraries
  \file{engine/libnsengine.a} and \file{libcompat/libregions.a}.
\end{itemize}

The header \file{engine/nonspec.h} contains the complete
nonspecialized \banshee{} API.

\section{Using the Engine Generator}
\label{sec-spec}

One of the major innovations in \banshee{} is
\emph{specialization}. \banshee{} includes a code generator that
specializes the constraint back-end for each program analysis. The
analysis designer describes a set of constructor signatures in a
specification file, which \banshee{} compiles into a specialized
constraint resolution engine. Specialization allows checking a host of
correctness conditions statically, and also improves software
maintenence. The insight here is that most program analyses only use a
fixed, static set of non-constant constructors. The \banshee{} code
generator essentially does partial evaluation of the constraint solver
with respect to the statically defined constructor signatures.

To use \banshee{} as a nonspecialized library, these steps should
be followed:

\begin{itemize}
\item The analysis designer writes a short \emph{specification}
  describing the signatures of constructors used in the
  analysis. Other than constants, only the constructors described in
  the specification will be available (new nonconstant constructors
  cannot be definedd dynamically).
\item The analysis designer runs the code generator
  \file{bin/banshee.exe} on their specification file. This generates a
  C source file and header file, containing the interfaces for the new
  specialized engine. 
\item The analysis designer includes the generated header file (and
  ideally no other \banshee{} headers) in the client application
\item The client application calls the appropriate initialization
  routine (a function whose name ends with \texttt{\_init}) before
  calling any other \banshee{} function
\item The analysis designer links their client with
  \file{engine/libengine.a}, the .o file produced by compiling the
  generated C source file, and the region library
  \file{libcompat/libregions.a}.
\end{itemize}

\subsection{Banshee Specifications}

Banshee specifications contain the same information as \ibanshee{}
constructor declarations, but the definitions are introduced
statically, before any constraints or expressions are built. We
illustrate the syntax with a simple example (from
\file{lambda/lambda.bsp}

\begin{verbatim}
specification lambda : LAMBDA =
 spec
  data l_type : term = boolean
		     | integer
	             | function of l_type * l_type
end
\end{verbatim}

In \ibanshee{} syntax, this specification would be roughly identical
to the following constructor definitions:

\begin{verbatim}
[0] > boolean : term
constructor: boolean
[0] > integer : term
constructor: integer
[0] > function(=term,=term) : term
constructor: function
\end{verbatim}

Note that the \banshee{} specification file is slightly more
fine-grained: the \ibanshee{} constructor defintions allow \emph{any}
expression of sort \texttt{term} to appear as subexpressions to the
\texttt{function} constructor. However, in the specification file,
only \texttt{l\_type} expressions can appear in the \texttt{function}
constructor. Conceptually, each \texttt{data} declaration defines a
new "type", whereas in \ibanshee{} (and the nonspecialized library)
each sort is a separate type.

Using a specialized engine has a number of advantages to using the
nonspecialized engine. Unless the client analysis requires a dynamic
set of non-constant constructors, we recommend specialization.

See \file{tests/lambda.bsp}, \file{tests/lambda-test.c} and the
associated \texttt{make} target in the \texttt{tests} directory for a
complete example.

\section{Using the Dyck CFL Reachability API}
\label{sec-dyckcfl}

Many program analyses can be reduced to a context free language
reachability problem over languages of matched parentheses (the
\emph{Dyck} languages). In \cite{kodumal:pldi04}, we describe an
efficient reduction from set constraints to Dyck CFL reachability. We
have implemented this reduction and have included it with \banshee{}.

The API for Dyck CFL reachability using \banshee{} can be found in
\file{dyckcfl/dyckcfl.h}. To use this API, the following steps should
be taken:

\begin{itemize}
\item The analysis designer should include the headers
  \file{engine/nonspec.h} and \file{dyckcfl/dyckcfl.h}, and ideally,
  no other \banshee{} headers.
\item Functions \texttt{nonspec\_init} and \texttt{dyck\_init} should
  be called first.
\item The analysis designer should link the libraries
  \file{engine/libnsengine.a} and \file{libcompat/libregions.a} along
  with the object file \file{dyckcfl/dyckcfl.o}.
\end{itemize}

See \file{tests/dyckcfl-test.c} and the associated \texttt{make}
target in the \texttt{tests} directory for a complete example.

\section{Reference}
\label{sec-reference}

\begin{comment}
\subsection{Sets in Inductive Form}
\label{sec-setif}

\subsection{Sets in Sub-Transitve Form}
\label{sec-setst}

\subsection{Terms}
\label{sec-term}

\subsection{Rows}
\label{sec-rows}
\end{comment}

\subsection{IBanshee Quick Reference}
\begin{verbatim}

ident : [A-Z a-z]([A-Z a-z 0-9 _])* integers (i) : [0-9]+

Variables (v)    : '{ident}
Constructors (c) : {ident}
Labels (l)       : {ident}
Names (n)        : {ident}

Expressions (e)  : v | c | n | c(e1,...,en) | e1 && e2 | e1 || e2
                 | <l1=e1,...,ln=en [| e]> | 0:s | 1:s | _:s
                 | pat(c,i,e) | proj(c,i,e) | ( e ) | c^-i(e)

sorts            : basesort | row(basesort)

basesort         : setIF | term

Var decl         : v : sort
Constructor decl : c(s1,...,sn) : basesort
Name decl        : n = e 
Sig (s)          : + sort | - sort | = sort

Constraints      : e1 <= e2 | e1 == e2

Commands         : !help
                   !tlb e
                   !ecr e
                   !undo [i]
                   !trace [i]
                   !quit
                   !save "filename"
                   !load "filename"
                   !rsave
                   !rload
                   !exit
\end{verbatim}

\subsection{Grammar for \banshee{} Specifications}

Specification files consist of a single \textit{specification}, drawn from
the following grammar:
$$
\begin{array}{lrl}
  \id{specification} &\kw{::=}& \kw{specification} ~\id{spcid} ~\kw{:}~ \id{hdrid} ~\kw{=} ~\kw{spec} ~\id{dataspec}~ \kw{end} \\
  \id{dataspec} & \kw{::=} & 
  \!\!\begin{array}[t]{rl}\kw{data} \!\!\! & \id{exprid}_1 ~\kw{:}~ \id{sort}_1 \bra{\id{sortopts}} 
               ~\bra{\kw{=} ~\id{conspec}_1}_1 \cdots \\ 
               \kw{and}\!\!\! & \id{exprid}_n ~\kw{:}~ \id{sort}_n \bra{\id{sortopts}} ~\bra{\kw{=} ~\id{conspec}_n}_n 
       \end{array} \\
  & | & \id{dataspec}_1 ~ \id{dataspec}_2 \\
  \id{conspec} & \kw{::=} & \id{conid} ~\bra{\kw{of} ~\id{consig}} \\
               & | & \id{conspec}_1~\kw{|} ~\id{conspec}_2 \\ 
 \id{consig} & \kw{::=} & \id{bconsig}_0~\kw{*} \cdots \kw{*}~\id{bconsig}_n \\
  \id{bconsig} & \kw{::=} & \id{vnc} ~\id{exprid} \\
  \id{vnc} & \kw{::=} & \kw{+} | \kw{-} | \kw{=} \\
  \id{sortopts} & \kw{::=} & \kw{[} \id{option_1} \kw{,} \cdots \kw{,}\id{option_n} \kw{]} \\
  \id{sort} & \kw{::=} & \kw{term} ~~|~~ \kw{setIF} ~~|~~ \kw{setST} ~~|~~ \kw{row(}\id{exprid}\kw{)}  \\ 
 

\end{array}
$$

Expression identifiers (\textit{exprid}), constructor identifiers 
(\textit{conid}), specification identifiers (\textit{spcid}), and header 
identifiers (\textit{hdrid}) must only be used once in a \texttt{.spec} file. 
Identifiers consist of an upper- or lowercase letter followed by a sequence 
of letters, digits, or underscores ('\texttt{\_}').

\bibliography{banshee-manual}\bibliographystyle{abbrv}

\appendix{}

\section{Known Bugs and Limitations}
\label{app-bugs}

\begin{itemize}

\item If backspace doesn't delete properly in \ibanshee{}, try typing
  \texttt{stty erase \char94?} at the shell prompt before invoking
  ibanshee.exe.

\item Projection merging and the DyckCFL API don't work properly (the
  bug has to do with group projections and merging). TURN OFF
  projection merging if using the DyckCFL API with globals. If you're
  not using globals, this won't matter, because group projections
  won't be used. Projection merging doesn't improve the DyckCFL
  library's performance, anyway.

\item Hashset (\file{hashset.c}) sometimes doesn't terminate. By
  default, banshee doesn't use hashset: it uses a bounds
  representation based on \file{hash.c} (\file{hash\_bounds.c}, as
  opposed to the hashset based \file{bounds.c}). However, the
  hashset-based bounds representation is about twice as fast as the
  hash-based bounds representation. If you want to try out the
  hashset-based bounds representation, do \texttt{make clean all check
    NO\_HASH\_BOUNDS=1}. Note that backtracking hasn't been
  implemented in hashset, so some of the check targets will fail.

\item Andersen's analysis requires pre-processed source files. We
  recommend pre-processing with gcc-2.95, as pre-processing with later
  versions of GCC may produce output that our C front-end cannot
  parse. Thanks to Stephen McCamant for this suggestion.

\end{itemize}

\section{Copyright}
\label{app-copyright}

This manual is copyright (C) 2005 The Regents of the University of
California.

\banshee{} is distributed under the BSD license, with the exception of
the following files:

\begin{itemize}
\item \file{engine/malloc.c}: see file for license
\item \file{codegen/OCamlMakefile}: see
  http://www.ai.univie.ac.at/~markus/home/ocaml\_sources.html for
  license
\item \file{cparser/*}: GPL, see cparser/COPYRIGHT. Note that the
  cparser is a separate application, no code from cparser is linked
  into \banshee{} itself.
\end{itemize}

See file \file{COPYRIGHT} for more details.

\end{document}
